\documentclass[addpoints,,12pt]{exam}

\title{Prova.2018.2.Linguagem.Programação.3}

\usepackage[utf8]{inputenc}
\usepackage[brazil]{babel}
\usepackage[T1]{fontenc}
\usepackage{mathtools}
\newcommand*{\renameenviron}[1]{%
  \expandafter\let\csname exam-#1\expandafter\endcsname
      \csname #1\endcsname
  \expandafter\let\csname endexam-#1\expandafter\endcsname
      \csname end#1\endcsname
  \expandafter\let\csname #1\endcsname\relax
  \expandafter\let\csname end#1\endcsname\relax
}
\renameenviron{framed}
\renameenviron{shaded}
\renameenviron{leftbar}
\usepackage{mdframed}
\usepackage{minted}

\mdfsetup{frametitlealignment=\centering}

\usepackage{enumerate}
\usepackage{paralist}
\usepackage{syntax}
\setdefaultleftmargin{0em}{2.2em}{1.87em}{1.7em}{1em}{1em}

\pointpoints{ponto}{pontos}

\newcommand{\figuratopo}{  
	\begin{center}
    	%\vbox {\includegraphics[width=0.20\textwidth]{figuras/logo_topo.jpeg}}
    	\vbox to 0pt{\hfill\fbox{\includegraphics[height=3cm]{figuras/qrcode.png}}}
    	%\includegraphics[scale=0.2]{figuras/qrcode.png}
  	\end{center}}
\renewcommand{\half}{
0.5
}

\begin{document}

%\figuratopo 
\pagestyle{headandfoot}
\firstpageheadrule
\firstpageheader{Prof. Nome do professor}{}{$2^{\text{o}}$ ano do curso - disciplina}
\runningheader{$2^{\text{o}}$ ano do curso - disciplina}{}{}
\runningheadrule
\runningfooter{}{}{p. \thepage\ - \numpages}

\begin{center}
	\bfseries \large Avaliação do $2^{\text{o}}$ Bimestre - 07/06/2018
\end{center}

\begin{mdframed}[userdefinedwidth=1.0\textwidth,align=center,frametitle={Instruções}]
			%\footnotesize
		\begin{enumerate}
			\item Escreva, abaixo, no local indicado o nome completo. 
			\item Esta prova é composta por \numpages\ páginas (incluindo a capa) e \numquestions\ questões.
			\item Aguarde autorização para iniciar a prova. A seguir, antes de iniciar a prova, confira a paginação e o número de questões.
			\item A interpretação das questões é parte do processo de avaliação, não sendo permitidas perguntas.
			\item Não serão permitidos empréstimos de materiais, consultas e comunicação entre os alunos, tampouco o uso de livros e apontamentos. Aparelhos eletrônicos e, em especial, aparelhos celulares deverão ser desligados.
			\item Utilizar folhas em branco para a solução das questões.
			\item As questões podem ser resolvidas e apresentadas fora de ordem, mas devem estar identificadas.
			\item Colocar nomes em todas as folhas utilizadas, inclusive nas que apresentam os enunciados das questões.
			\item Todas as folhas utilizadas devem ser entregues.
			\item Questões resolvidas à lápis podem causar problemas em uma possível revisão de prova. Portanto, prefira respostas a caneta nas partes importantes da solução das questões.
		\end{enumerate}

	\hfill \textbf{Boa Prova!}
	\end{mdframed}
	
	\vfill
	
	\begin{center}
		\textbf{Campo reservado ao professor} \\
		\hqword{Questão}
		\hpword{Pontos}
		\hsword{Nota}
		\gradetable[h]%[pages]  % Use [pages] to have grading table by page instead of question	
	\end{center}
	\vspace{.5cm}
	
	\begin{mdframed}[align=left]
		\textbf{Nome: 11 - NOME COMPLETO DO ALUNO 11}
		\\
		\\Assinatura:
	\end{mdframed}

	\newpage	
	
\noindent \textbf{Questões}

\begin{questions}
\question[10]

Os lexemas de uma linguagem de programação não incluem:

\begin{choices}
\item Literais numéricos
\item Operadores
\item Identificadores
\item Palavras reservadas
\item Variáveis %verdadeiro
\end{choices}
\answerline


\question[10]

Identifique abaixo as razões pelas quais os analisadores sintáticos são
baseados em gramáticas.

I) Item I\\
II) Item II.\\
III) Item III.\\

Dos itens listados acima, quais são verdadeiros?\\

\begin{choices}
\item nenhum dos itens
\item I e III apenas
\item todos os itens %verdadeiro
\item II e III apenas
\item I apenas
\end{choices}
\answerline



\question[10]
Analise as frases abaixo e assinale a alternativa que é verdadeira.\\
\\
\begin{choices}
\item TEXTO DA ALTERNATIVA.
\item TEXTO DA ALTERNATIVA.
\item TEXTO DA ALTERNATIVA.
\item TEXTO DA ALTERNATIVA.
\item TEXTO DA ALTERNATIVA. %verdadeiro
\end{choices}
\answerline


\question[10]
Encontre na lista abaixo exemplos de reconhecedores de linguagens:\\
\\
I. Item I\\
II. Item II\\
III. Item III\\
IV. Item IV
\\
\begin{choices}
\item III e IV apenas
\item II e III apenas
\item I e II apenas
\item Nenhum %verdadeiro
\item Todos
\end{choices}
\answerline


\question[10]

Os lexemas de uma linguagem de programação não incluem:

\begin{choices}
\item Variáveis %verdadeiro
\item Identificadores
\item Palavras reservadas
\item Literais numéricos
\item Operadores
\end{choices}
\answerline


\question[10]

Os lexemas de uma linguagem de programação não incluem:

\begin{choices}
\item Operadores
\item Variáveis %verdadeiro
\item Palavras reservadas
\item Identificadores
\item Literais numéricos
\end{choices}
\answerline


\question[10]

Os lexemas de uma linguagem de programação não incluem:

\begin{choices}
\item Operadores
\item Variáveis %verdadeiro
\item Palavras reservadas
\item Identificadores
\item Literais numéricos
\end{choices}
\answerline


\question[10]
Encontre na lista abaixo exemplos de reconhecedores de linguagens:\\
\\
I. Item I\\
II. Item II\\
III. Item III\\
IV. Item IV
\\
\begin{choices}
\item III e IV apenas
\item II e III apenas
\item I e II apenas
\item Nenhum %verdadeiro
\item Todos
\end{choices}
\answerline


\question[10]

Os lexemas de uma linguagem de programação não incluem:

\begin{choices}
\item Variáveis %verdadeiro
\item Identificadores
\item Palavras reservadas
\item Literais numéricos
\item Operadores
\end{choices}
\answerline


\question[10]
Dada a gramática abaixo, quais das produções passam pelo teste de disjunção?\\
\\
\begin{tabular}{|l c l}
	S & $\rightarrow$ & aSb | bAA \\
    S & $\rightarrow$ & b{aB} | a \\
	L & $\rightarrow$ & aB | a \\
\end{tabular}

De acordo com cada não terminal da gramática:
\begin{choices}
\item S - passa, A - passa, B - passa
\item S - passa, A - passa, B - falha %verdadeiro
\item S - falha, A - falha, B - falha
\item S - passa, A - falha, B - passa
\item S - falha, A - falha, B - passa
\end{choices}
\answerline


\end{questions}

\end{document}
